\subsubsection*{GPS}

The Global Positioning System (GPS), originally Navstar GPS, is a satellite-based radio navigation system owned by the United States government and operated by the United States Air Force.[2] It is a global navigation satellite system that provides geolocation and time information to a GPS receiver anywhere on or near the Earth where there is an unobstructed line of sight to four or more GPS satellites.

\subsubsection*{AR}

Augmented reality is the integration of digital information with the user's environment in real time. Unlike virtual reality, which creates a totally artificial environment, augmented reality uses the existing environment and overlays new information on top of it.

\subsubsection*{API}

A set of routine definitions, protocols, and tools for building software and applications. An API specification can take many forms, but often include specifications for routines, data structures, object classes, or variables.

\subsubsection*{GUI)}

A graphical user interface utilizes the graphical capabilities of a computer to provide a
user interface for computer software. GUIs often employ images, symbols and visual cues
to facilitate user interaction in a point-and-click manner. Common methods for providing
user input include the use of keyboards, mice, pen devices, touch screens or other devices
for physical interaction.

\subsubsection*{Open Source Software}

Open Source Software (OSS) is software distributed with a license allowing access to its
source code, free redistribution, the creation of derived works, and unrestricted use.

\subsubsection*{LALR}

Look-Ahead Left to Right. LALR parsing algorithm, introduced by Frank DeRemer, provides the same high performance of LR parsing algorithm, introduced by Donald Knuth, but is more efficient in term of size.

\subsubsection*{LaTeX}

LaTeX is a document markup language and document preparation systems for the TeX typesetting program.

\subsubsection*{SUT}

System Under Test. This term refers to a system being tested for correct operation.

